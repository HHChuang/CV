\section{\sc Teaching Experience}
% {\bf{Copmany Name}}\\
% \begin{tabular}{@{}p{4in}p{2in}}
% Title & Dates\\
% \end{tabular}
% \begin{itemize}
% \setlength\itemsep{0em}
% \item Description
% \item Description
% \item Description
% \end{itemize}
% \begin{tabular}{@{}p{4in}p{2in}}
% Additional Title & Dates\\
% Additional Title & Dates\\
% \end{tabular}

{\bf  International Max Planck Research School } \hfill \textit{Aug. 2025} \\
Exploring and Harnessing Non-Adiabatic Processes\hfill Germany
\begin{itemize}
    \setlength\itemsep{0em}
     \item Designed and delivered a three-hour lecture on Non-Adiabatic Molecular Dynamics (NAMD), using Mathematica to demonstrate three basic methods: Born–Oppenheimer Molecular Dynamics (BOMD), Ehrenfest dynamics, and Trajectory Surface Hopping (TSH).
    \item Combined analytical models with electronic structure theory (EST) concepts to illustrate the role of EST and energy near-degeneracy regions—such as avoided crossings—in model diatomic molecular systems.
\end{itemize}
\vspace{-2.2em} \hfill \href{https://www.imprs-pks.mpg.de/events/schools-and-workshops}{[IMPRS Summer School 2025]}

{\bf University of Leeds } \hfill \textit{Oct. 2022} \\
Workshop for Physical Chemists \hfill UK
\begin{itemize}
    \setlength\itemsep{0em}
    \item Delivered specialised lectures on electronic structure theory (EST), with a rigorous theoretical derivation of EST models tailored to physical chemists.
    \item Emphasised the mathematical foundations and practical applications of quantum chemistry models, ensuring students grasped both conceptual and real-world aspects.
\end{itemize}

{\bf University of Bristol}  \hfill \textit{June 2021} \\
Workshop for Mathematicians  \hfill UK
\begin{itemize}
    \setlength\itemsep{0em}
    \item Designed and taught lectures on simplified Hartree-Fock theory, adapted for audiences with a formal mathematical background.
    \item Used a Jupyter Notebook-based implementation of the helium atom to illustrate computational workflows in EST.
    \item Promoted an understanding of numerical techniques underpinning quantum models via hands-on Python coding. 
\end{itemize}
\vspace{-2.2em} \hfill \href{https://github.com/HHChuang/HartreeFockPractice/blob/master/src/He_atom.ipynb}{[Teaching code on GitHub]}

{\bf University of California, Davis}  \hfill \textit{Aug. 2018 -- Oct. 2018} \\
Workshop for Organic Chemists \hfill USA
\begin{itemize}
    \item Conducted chalkboard lectures connecting electronic structure theory with practical applications in organic reaction mechanisms.
    \item Integrated examples from students' research to demonstrate how computational methods elucidate reaction pathways and molecular properties.
    \item Developed and provided mathematical exercises aligned with their chemical interests to foster interdisciplinary learning.
\end{itemize}

{\bf  Chung-Yuan Christian University } \hfill   \textit{Sept. 2007 -- Feb. 2008} \\
Teaching Assistant of Quantum Chemistry \hfill Taiwan
\begin{itemize}
    \setlength\itemsep{0em}
    \item Supported undergraduate teaching by leading tutorials, explaining theoretical concepts, and answering student questions during office hours.
    \item Reinforced students' understanding of quantum mechanics fundamentals through applied problem-solving.
\end{itemize}

\section{\sc Advising Experience}
{\bf  School of Chemistry, University of Leeds } \hfill   \textit{Feb. 2022 -- Feb. 2023} \\
\null \hfill UK
\begin{itemize}
    \setlength\itemsep{0em}
    \item  Supervised a first-year PhD student, offering guidance on her research project on the coupled-coherent state method and supporting her academic development.
    \item Held regular one-on-one meetings to discuss research progress and designed targeted tasks to strengthen her mathematical and programming skills.
\end{itemize}
\endinput