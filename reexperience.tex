\section{\sc Research Experience}

% Pros:
% 	•	Outstanding depth and progression.
% 	•	Clear descriptions of responsibilities and methodologies.
% 	•	Global experience across leading institutions.

% Cons:
% 	•	Slight repetition in responsibilities (e.g., “constructed PES”, “studied reactions”).
% 	•	Some terms (like “phase space structure”) could be inaccessible to broader readers.

% Suggestions:
% 	•	Condense early-stage research (pre-PhD) into a short list.
% 	•	Include impact: results, published outcomes, collaborations.


{\bf Guest Scientist} \hfill \textit{Feb. 2023 -- Present} \\
Finite System Department  \hfill Dresden, Germany \\
Max Planck Institute for the Physics of Complex Systems (MPIPKS) \\
Principal Investigator: Dr. Alexander Eisfeld\\
\vspace{-10pt}
\begin{itemize}
\setlength\itemsep{0em}
    \item Investigated the impact of transition current density in the strong-coupling regime (coherent dynamics) for confined 2D materials to enhance the understanding of excitonic interactions.
\end{itemize}

{\sl \bf Research Associate} \hfill \textit{Aug. 2021 -- Feb. 2023} \\
School of Chemistry, University of Leeds \hfill Leeds, UK\\
Principal Investigator: Dr. Dmitry Shalashilin \\
\vspace{-10pt}
\begin{itemize}
    % (\href{https://arxiv.org/abs/2306.07407}{Arxiv 2023}). 
    \item Studied the reaction dynamics of electron-impact processes using conventional trajectory-based methods in the field of non-adiabatic molecular dynamics.
\end{itemize}
%}

{\sl \bf Research Associate} \hfill \textit{Dec. 2020 -- July 2021} \\
%{
School of Applied Mathematics, University of Bristol \hfill Bristol, UK\\
Principal Investigator: Dr. Stephen R. Wiggins \\
\vspace{-10pt}
\begin{itemize}
    \item Applied Lagrangian descriptors to analyse phase space structures in general reaction dynamics.
\end{itemize}

{\sl \bf Postdoctoral Researcher} \hfill \textit{Sept. 2020 -- Oct. 2020}\\
%{
Institute of Chemistry, Academia Sinica \hfill Taipei, Taiwan \\
Principal Investigator: Dr. Chao-Ping Hsu \\
\vspace{-10pt}
\begin{itemize}
    \item Investigated electronic correlation effects in interstellar glycine formation using high-level electronic excited-state structure methods.
\end{itemize}
%}

{\sl \bf Visiting Scholar} \hfill \textit{May 2018 -- Apr. 2019} \\
%{
Department of Chemistry, University of California, Davis \hfill California, USA\\
Supervisor: Dr. Dean J. Tantillo \\
\vspace{-10pt}
\begin{itemize} %\itemsep -2pt
    \item Investigated dynamics and and built tools to generate potential energy surfaces for reactions with post-transition state bifurcation.
\end{itemize}
%}

{\sl \bf Doctoral Research} \hfill \textit{Aug. 2015 -- Apr. 2018} \\
%{
Institute of Chemistry, Academia Sinica \hfill Taipei, Taiwan \\
Supervisor: Dr. Chao-Ping (Cherri) Hsu \\
\vspace{-10pt}
\begin{itemize} %\itemsep -2pt
    \item Used Q-Chem and diabatic Hamiltonians for constructing 
    potential energy curves of proton-coupled electron transfer (PCET) reactions.
    \item Studied sugar chemistry with experimentalists.
\end{itemize}
%}

{\sl \bf Doctoral Research}  \hfill \textit{Aug. 2013 -- Jul. 2015} \\
%{
Institute of Atomic and Molecular Sciences, Academia Sinica \hfill Taipei, Taiwan \\
Supervisor: Dr. Jer-Lai Kuo \\

\vspace{-10pt}
\begin{itemize} %\itemsep -2pt
    \item Derived the analytical expression up to the 4$^{th}$-body term of anharmonic oscillators.
\end{itemize}
%}

{\sl \bf Research Assistant} \hfill \textit{Aug. 2012 --  Jul. 2013} \\
%{
Institute of Atomic and Molecular Science, Academia Sinica \hfill Taipei, Taiwan \\
Supervisor: Dr. Jer-Lai Kuo \\

\vspace{-10pt}
\begin{itemize} %\itemsep -2pt
    \item Understand the anharmonicity of the \ch{O-H} bond in molecular clusters using IR spectrum for their predissociation reactions.
\end{itemize}
%}

{\sl \bf Masters Research} \hfill \textit{Aug. 2010 -- Jul. 2012} \\
%{
Department of Chemistry and Biochemistry \hfill Chiayi, Taiwan \\
National Chung-Cheng University \\
Supervisor: Dr. Wei-Ping Hu \\
\vspace{-10pt}
\begin{itemize} %\itemsep -2pt
    \item Found the barrierless intramolecular proton transfer mechanism and confirmed by time-resolved femtosecond spectroscopy.
\end{itemize}


\endinput